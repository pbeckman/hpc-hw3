\documentclass{article}
\usepackage{
  amsmath, amsthm, amssymb, amsfonts, 
  mathtools, xfrac, dsfont, mathrsfs, bm
  }
\usepackage{hyperref, float, parskip}
\usepackage[margin=0.8in]{geometry}
\usepackage[justification=centering,labelfont=bf]{caption}
\renewcommand{\arraystretch}{1.3}

% grouping and bookending
\newcommand{\pr}[1]{\left(#1\right)}
\newcommand{\br}[1]{\left[#1\right]}
\newcommand{\cbr}[1]{\left\{#1\right\}}
\newcommand{\floor}[1]{\left\lfloor#1\right\rfloor}
\newcommand{\ceil}[1]{\left\lceil#1\right\rceil}
\newcommand{\abs}[1]{\left|#1\right|}
\newcommand{\norm}[1]{\left\lVert#1\right\rVert}
\newcommand{\ip}[1]{\left\langle#1\right\rangle}
\renewcommand{\vec}[1]{\left\langle#1\right\rangle}
% derivatives
\newcommand{\der}[2]{\frac{d #1}{d #2}}
\newcommand{\mder}[2]{\frac{D #1}{D #2}}
\newcommand{\pder}[2]{\frac{\partial #1}{\partial #2}}
% common bold and script letters
\newcommand{\C}{\mathbb{C}}
\newcommand{\E}{\mathbb{E}}
\newcommand{\F}{\mathcal{F}}
\newcommand{\G}{\mathcal{G}}
\renewcommand{\L}{\mathscr{L}}
\newcommand{\N}{\mathbb{N}}
\renewcommand{\O}{\mathcal{O}}
\renewcommand{\P}{\mathbb{P}}
\newcommand{\Q}{\mathbb{Q}}
\newcommand{\R}{\mathbb{R}}
\renewcommand{\S}{\mathbb{S}}
\newcommand{\Z}{\mathbb{Z}}
% math operators
\DeclareMathOperator{\Cov}{Cov}
\DeclareMathOperator{\Var}{Var}
\let\Re\relax
\DeclareMathOperator{\Re}{Re}
\let\Im\relax
\DeclareMathOperator{\Im}{Im}
\DeclareMathOperator{\diag}{diag}
\DeclareMathOperator{\tr}{tr}
\DeclareMathOperator*{\argmax}{arg\,max}
\DeclareMathOperator*{\argmin}{arg\,min}
% misc
\newcommand{\mat}[1]{\begin{bmatrix}#1\end{bmatrix}}
\newcommand{\ind}[1]{\mathds{1}_{#1}}
\renewcommand{\epsilon}{\varepsilon}

\setlength\parindent{0pt}

\title{High Performance Computing: Homework 3}
\author{Paul Beckman}
\date{}

\begin{document}

\maketitle

\section{Pitch your final project}
See email.

\section{Approximating Special Functions Using Taylor Series \& Vectorization}
Adding additional terms into the Taylor expansion in \texttt{sin4\_vector} gives 
\begin{verbatim}
Reference time: 26.5879
Taylor time:    5.9070      Error: 6.928125e-12
Intrin time:    1.7174      Error: 2.454130e-03
Vector time:    2.8851      Error: 6.928125e-12
\end{verbatim}

\section{Parallel Scan in OpenMP}
After increasing \texttt{N} by an order of magnitude to get more consistent runtimes, on an Intel(R) Xeon(R) CPU @ 2.53GHz with 8 cores we obtain the following runtimes for our parallel implementation.
\begin{table}[H]
  \centering
  \begin{tabular}{c|c}
    \texttt{OMP\_NUM\_THREADS} & \textbf{time (s)} \\
    \hline
    2 & 5.29 \\
    4 & 5.29 \\
    8 & 5.99 \\
    16 & 3.84 \\
    32 & 2.91 \\
    64 & 2.91 
  \end{tabular}
\end{table}


\end{document}
